% ab-tx.tex

\documentclass[12pt, letterpaper]{article}

% preamble.tex

%% added by hfwei %%
\usepackage[svgnames]{xcolor}

\usepackage[framemethod=tikz]{mdframed}
\newcommand{\bib}[1]{
  \begin{mdframed}[hidealllines = true, backgroundcolor = blue!20]
    #1
  \end{mdframed}
}
\newcommand{\note}[1]{
  \begin{mdframed}[hidealllines = true, backgroundcolor = purple!20]
    #1
  \end{mdframed}
}
%% added by hfwei %%

% just for the example
\usepackage{lipsum}
% Set margins to 1.5in
\usepackage[margin=1.5in]{geometry}

\usepackage{graphicx}

% for crimson text
\usepackage{crimson}
\usepackage[T1]{fontenc}

% setup parameter indentation
\setlength{\parindent}{0pt}
\setlength{\parskip}{6pt}

% for 1.15 spacing between text
\renewcommand{\baselinestretch}{1.15}

% For defining spacing between headers
\usepackage{titlesec}
% Level 1
\titleformat{\section}
  {\normalfont\fontsize{18}{0}\bfseries}{\thesection}{1em}{}
% Level 2
\titleformat{\subsection}
  {\normalfont\fontsize{14}{0}\bfseries}{\thesection}{1em}{}
% Level 3
\titleformat{\subsubsection}
  {\normalfont\fontsize{12}{0}\bfseries}{\thesection}{1em}{}
% Level 4
\titleformat{\paragraph}
  {\normalfont\fontsize{12}{0}\bfseries\itshape}{\theparagraph}{1em}{}
% Level 5
\titleformat{\subparagraph}
  {\normalfont\fontsize{12}{0}\itshape}{\theparagraph}{1em}{}
% Level 6
\makeatletter
\newcounter{subsubparagraph}[subparagraph]
\renewcommand\thesubsubparagraph{%
  \thesubparagraph.\@arabic\c@subsubparagraph}
\newcommand\subsubparagraph{%
  \@startsection{subsubparagraph}    % counter
    {6}                              % level
    {\parindent}                     % indent
    {12pt} % beforeskip
    {6pt}                           % afterskip
    {\normalfont\fontsize{12}{0}}}
\newcommand\l@subsubparagraph{\@dottedtocline{6}{10em}{5em}}
\newcommand{\subsubparagraphmark}[1]{}
\makeatother
\titlespacing*{\section}{0pt}{12pt}{6pt}
\titlespacing*{\subsection}{0pt}{12pt}{6pt}
\titlespacing*{\subsubsection}{0pt}{12pt}{6pt}
\titlespacing*{\paragraph}{0pt}{12pt}{6pt}
\titlespacing*{\subparagraph}{0pt}{12pt}{6pt}
\titlespacing*{\subsubparagraph}{0pt}{12pt}{6pt}

% Set caption to correct size and location
\usepackage[tableposition=top, figureposition=bottom, font=footnotesize, labelfont=bf]{caption}

% set page number location
\usepackage{fancyhdr}
\fancyhf{} % clear all header and footers
\renewcommand{\headrulewidth}{0pt} % remove the header rule
\rhead{\thepage}
\pagestyle{fancy}

% Overwrite Title
\makeatletter
\renewcommand{\maketitle}{\bgroup
   \begin{center}
   \textbf{{\fontsize{18pt}{20}\selectfont \@title}}\\
   \vspace{10pt}
   {\fontsize{12pt}{0}\selectfont \@author} 
   \end{center}
}
\makeatother

% Used for Tables and Figures
\usepackage{float}

% For using lists
\usepackage{enumitem}

% For full citations inline
\usepackage{bibentry}
\nobibliography*
% bibentry with colored background box
\newcommand{\cbibentry}[1]{\colorbox{Gainsboro!60!Lavender}{\bibentry{#1}}}

% Custom Quote
\newenvironment{myquote}[1]%
  {\list{}{\leftmargin=#1\rightmargin=#1}\item[]}%
  {\endlist}
  
% Abstract 
\renewenvironment{abstract}{
  \vspace*{-.5in}\fontsize{12pt}{12}
  \begin{myquote}{.5in}
    \noindent \par{\bfseries \abstractname.}}
    {\medskip\noindent
  \end{myquote}}%

\author{Hengfeng Wei \\ hfwei@nju.edu.cn}
\title{Annotated Bibliography on Transactions}
%%%%%%%%%%%%%%%%%%%%%%%%%%%%%%%%%%%%%%%%
\begin{document}

\maketitle
\thispagestyle{fancy}
%%%%%%%%%%%%%%%%%%%%%%%%%%%%%%
\section{Books} \label{section:books}

\bib{\bibentry{TIS:Book2001}}
\note{\it 
  It is researchers-oriented. Highly recommended.
}
%%%%%%%%%%%%%%%%%%%%%%%%%%%%%%
\section{Transactional Consistency Models} \label{section:tcm}

%%%%%%%%%%%%%%%%%%%%
\subsection{Frameworks} \label{ss:framework}

\bib{\bibentry{CritiqueIsolation:SIGMOD1995}}
\note{\it
  Defines Isolation Levels in terms of phenomena;
  Introduces new phenomena;
  Define Snapshot Isolation.
}

\bib{\bibentry{IsolationHistory:IST1997}}
\note{\it 
  This paper formulates these different degrees of isolation in terms of histories, 
  as in the case of the usual serialization theory 
  and proposes timestamp-based protocols for different degrees of isolation.
}

\bib{\bibentry{PushPull:PLDI2015}}
\note{\it
  We present a general theory of serializability, 
  unifying a wide range of transactional algorithms, 
  including some that are yet to come.
  To this end, we provide a compact semantics in which 
  concurrent transactions PUSH their effects into the shared view 
  (or UNPUSH to recall effects) 
  and PULL the effects of potentially uncommitted concurrent transactions 
  into their local view (or UNPULL to detangle).
}

\bib{\bibentry{Crooks:PODC2017}}
\note{\it 
  This paper introduces the first state-based formalization of isolation guarantees.
}

\bib{\bibentry{ClientCentric:PhDThesis2019}}
\note{\it
  The PhD Thesis version of~\cite{Crooks:PODC2017}.
}
%%%%%%%%%%%%%%%%%%%%
\subsection{Serializability} \label{ss:sr}

\bib{\bibentry{PL:CACM1976}}
\note{\it 
  This is the first paper to formalize mathematically the concurrency control problem.
  It also defines ``conflict serializability'', 
  which is termed DSR in~\cite{Papadimitriou:JACM1979}.
}

\bib{\bibentry{Papadimitriou:JACM1979}}
\note{\it
  It is shown that recognizing the transaction histories that are serializable
  is an NP-complete problem.
  Several efficiently recognizable subclasses are introduced.
}

\bib{\bibentry{DistributedLocking:PODS1982}}
\note{\it
  We examine the problem of determining whether a set of locked transactions, 
  accessing a distributed database, is guaranteed to produce only serializable schedules. 
  For a pair of transactions we prove that this concurrency control problem 
  (which is polynomially solvable for centralized databases) 
  is in general coNP-complete.
}

\bib{\bibentry{SerializabilityLocking:JACM1984}}
\note{\it
  It is shown that locking cannot achieve the full power of serializability.
  An exact characterization of the schedules that can be produced 
  if locking is used to control concurrency is given for two versions of serializability:
  state serializability and view serializability.

  See also its STOC conference version~\cite{SerializabilityLocking:STOC1981}.
}

\bib{\bibentry{OneCopySR:TSE1984}}
\note{\it
  Introduce One Copy Serializability (1SR), 
  as a distributed/replicated counterpart of Serializability in a single-server system.
}

\bib{\bibentry{DCC:SIAM1985}}
\note{\it 
  We present a formal framework for distributed databases, 
  and we study the complexity of the concurrency control problem in this framework.
  Our transactions are partially ordered sets of actions, 
  as opposed to the straight-line programs of the centralized case.
  The concurrency control algorithm, or scheduler, is itself a distributed program.
}

\bib{\bibentry{SSI:VLDB2012}}
\note{\it 
  This paper describes our experience implementing PostgreSQL’s 
  new serializable isolation level.
  It is based on the recently-developed Serializable Snapshot Isolation (SSI) technique.
  This is the first implementation of SSI in a production database release 
  as well as the first in a database that did not previously 
  have a lock-based serializable isolation level. 
}
%%%%%%%%%%%%%%%%%%%%
\subsection{Snapshot Isolation} \label{ss:si}

\bib{\bibentry{CritiqueSI:EuroSys2012}}
\note{\it 
  We introduce write-snapshot isolation, a novel isolation level 
  that has a performance comparable with that of snapshot isolation, 
  and yet provides serializability. 
  The main insight in write-snapshot isolation is 
  to prevent read-write conflicts in contrast to write-write conflicts 
  that are prevented by snapshot isolation.
}
%%%%%%%%%%%%%%%%%%%%%%%%%%%%%%
\section{Theory} \label{section:theory}

%%%%%%%%%%%%%%%%%%%%%%%%%%%%%%
\section{Robustness (and Dependency Graphs)} \label{section:robustness}

\bib{\bibentry{Fekete:TODS2005}}
\note{\it
  This article develops a theory that characterizes 
  when nonserializable executions of applications can occur under SI.
}
%%%%%%%%%%%%%%%%%%%%%%%%%%%%%%
\section{Concurrency Control Protocols} \label{section:cc}

%%%%%%%%%%%%%%%%%%%%
\subsection{Overview} \label{ss:cc-overview}

\bib{\bibentry{CC:CSUR1981}}
\note{\it 
  In this paper we survey, consolidate, and present the state of the art 
  in distributed database concurrency control.
  The heart of our analysts is a decomposition of the concurrency control problem 
  into two major subproblems: read-write and write-write synchronization.
  We describe a series of synchronization techniques for solving each subproblem 
  and show how to combine these techniques into algorithms 
  for solving the entire concurrency control problem.
  Such algorithms are called ``concurrency control methods.''
  We describe 48 principal methods, including all practical algorithms 
  that have appeared m the literature plus several new ones.
  We concentrate on the structure and correctness of concurrency control algorithms.
  Issues of performance are given only secondary treatment.
}
%%%%%%%%%%%%%%%%%%%%
\subsection{Locking} \label{ss:locking}

\bib{\bibentry{Beyond2PL:JACM1985}}
\note{\it
  Graph protocols.
}

\bib{\bibentry{CCLocking:CSUR1998}}
\note{\it 
  This tutorial reviews CC methods based on standard locking, 
  restart-oriented locking methods, two-phase processing methods 
  including optimistic CC, and hybrid methods 
  (combining optimistic CC and locking) in centralized systems.
}
%%%%%%%%%%%%%%%%%%%%%%%%%%%%%%
\section{Formal Methods} \label{section:formal-methods}

\bib{\bibentry{ProgramLogic2PL:TR2017}}
\note{\it 
  We present a program logic for serializable transactions 
  that are able to manipulate a shared storage. 

   We show this by providing the first application of our logic 
   in terms of the Two-phase locking (2pl) protocol which ensures serializability.
}
%%%%%%%%%%%%%%%%%%%%%%%%%%%%%%
\section{Systems} \label{section:systems}

%%%%%%%%%%%%%%%%%%%%%%%%%%%%%%
\bibliographystyle{../acm-sigchi} 
\nobibliography{ab-tx}

\end{document}
%%%%%%%%%%%%%%%%%%%%%%%%%%%%%%%%%%%%%%%%